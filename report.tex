\documentclass[12pt, a4paper]{article} % Assuming 'article' class, adjust if needed

% --- Preamble (Include necessary packages) ---
\usepackage[utf8]{inputenc}
\usepackage[T1]{fontenc}
\usepackage[english]{babel}
\usepackage[a4paper, margin=2.5cm]{geometry} % Adjust margins as needed
\usepackage{graphicx}
\usepackage{hyperref}
\usepackage{amsmath, amssymb}
\usepackage{booktabs}
\usepackage{caption}
\usepackage{float} % For [H] placement if needed later
\usepackage{csquotes} % Required by biblatex when using babel
\usepackage[backend=bibtex, style=numeric]{biblatex} % changed backend to bibtex for simpler compile (or keep biber if you prefer)
\addbibresource{references.bib} % use plain relative filename; ensure references.bib is in the same folder as report.tex

% --- ADD: Allow underscores in normal text to avoid "Missing $ inserted" errors ---
\usepackage{underscore}

\hypersetup{
    colorlinks=true,
    linkcolor=blue,
    filecolor=magenta,      
    urlcolor=cyan,
    pdftitle={Report_Delaplace_20245433},
    pdfpagemode=FullScreen
}

% --- Custom Title Page Info ---
\newcommand{\coursename}{Module: Algorithms and Combinatorial Thinking}
\newcommand{\academicyear}{2025-2026}
\newcommand{\instructorname}{Dr. Franck Delaplace}
\newcommand{\groupmembers}{Christos Christophoros Mitsakopoulos}

\begin{document}

% --- Title Page ---
\begin{titlepage}
    \centering % Center everything on the title page

    % --- Logos ---
    \begin{minipage}[t]{0.1\textwidth}
        \centering
        \includegraphics[height=2.5cm]{logo_blue.png} % Adjust height as needed
    \end{minipage}
    \hfill % Pushes logos to opposite sides
    \begin{minipage}[t]{0.25\textwidth}
        \centering
        \raisebox{0.7cm}{\includegraphics[height=1.5cm]{logo_purple.png}} % Adjust height as needed
    \end{minipage}

    
    {\Huge \bfseries Metaheuristic approach to the Hamiltonian Path \par}


    \vspace{1cm} % Space between logos and title
    \begin{minipage}[t]{1\textwidth}
        \centering
        \includegraphics[width=12cm]{title_page.png}
        \captionof{figure}{\textit{Visualised with Pyvis, depicts the correct Hamiltonian path through the artificially generated, 50 node graph.}}
    \end{minipage}
    \vspace{0.3cm}

    % --- Course/Module Info ---
    {\Large \coursename \space \academicyear \par}

    \vspace{1cm} % Space before author/group info

    % --- Author/Group Info ---
    {\large \textbf{Written by:} \par}
    \vspace{0.2cm}
    \begin{Large}
    \groupmembers
    \end{Large}
    \par

    \vspace{0.5cm} % Space before instructor info

    % --- Instructor Info ---
    {\large \textbf{Supervised by:} \space \instructorname \par}


    \vfill % Pushes the date to the bottom

    % --- Date ---
    {\large \today \par} % Automatically inserts the compilation date
    % Or use: {\large October 27, 2025 \par} for a fixed date

    %\vspace*{1cm} % Add some space at the bottom

\end{titlepage}

% --- Table of Contents ---
\tableofcontents
\newpage

% --- Introduction ---
\section{Abstract}

Given a graph with a set of vertices, $V$, a path, $G$, is described as a Hamiltonian path, given that it passes through each vertex of a graph exactly once. Importantly, there is no efficient polynomial-time algorithm known to solve it for general graphs. As $V$ increases, the computational time nedded to solve it by brute force grows factorially. As such, this necessitates the use of metaheuristics for larger instances (=graphs). Importantly, the Hamiltonian path is critical for de novo genome assembly, as arranging reads in an acyclic graph -- one seeks to maximise the reads used within the a single path, to create a contiguous sequence. In this report a variety of metaheuristic approaches are tested against each other and a smaller implementation of a brute force search. The approaches include: \textit{Simulated Annealing}, \textit{Tabu Search} and a \textit{Genetic Algorithm} (appropriated from class), compared against a simple brute force approach for demonstration purposes.

\bigskip

Find all relevant results / as well as reproduce the experiment using the \texttt{christos_mitsakopoulos.ipynb} -- Jupyter Notebook, clone the repository at the following address for additional proof of work: \url{https://github.com/cmitsakopoulos/Delaplace_coursework}

\section{Test Environment: Erdős-Rényi Random Graph}

In order to benchmark the metaheuristic approaches in this report, an Erdős-Rényi (ER) $G(n,p)$ model was used. Where: $n$ is the number of vertices / nodes of the ER graph, while $p$ the probability an edge exists between two distinct nodes in that graph. Increasing the parameter $p$ influences the edge-density of this unstructured graph type, making it less challenging to identify a Hamiltonian path. This phenomenon became apparent when testing the configuration within my \texttt{christos_mitsakopoulos.ipynb} testing environment; for instance, a $p$ of $\approx0.3$ saw all metaheuristic approaches converge to a perfect "score" -- zero broken edges (=nodes uncovered). An expected outcome given the abundance of edges between any node in the graph, and by effect multiple Hamiltonian paths for the metaheuristic algorithms to identify.

\bigskip

As you might observe in the Jupyter Notebook, the solution of each algorithm is encoded as a permutation vector $S = [v_1, v_2, v_3, ..., v_n]$. Expectedly, the constraint here is that each $v_i$ (where $i\in n$), must be present only once in $S$ -- such that it represents a Hamiltonian Path ($G$).
 
\end{document}